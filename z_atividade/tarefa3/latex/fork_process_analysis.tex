\documentclass[a4paper,12pt]{article}

\usepackage[utf8]{inputenc}
\usepackage[T1]{fontenc}
\usepackage{lmodern}
\usepackage[brazilian]{babel}
\usepackage{amsmath, amssymb}
\usepackage{graphicx}
\usepackage{booktabs}
\usepackage{xcolor}
\usepackage{geometry}
\usepackage{float}
\usepackage{listings}
\usepackage{hyperref}

\geometry{margin=2.5cm}

\title{Análise de Hierarquia de Processos com Fork}
\author{Projeto da Tarefa 3}
\date{\today}

\begin{document}

\maketitle

\section{Introdução}

Este relatório apresenta uma análise detalhada do comportamento de processos em um sistema Unix, focando especificamente na criação de hierarquias de processos usando a chamada de sistema \texttt{fork()} e nas condições que levam à criação de processos órfãos.

Processos órfãos são aqueles cujo processo pai terminou antes deles. Nestes casos, o processo órfão é adotado pelo processo \texttt{init} (PID 1), que se torna seu novo pai. Esta característica é fundamental para entender o comportamento e a estabilidade de sistemas com múltiplos processos.

\section{Descrição dos Experimentos}

Foram realizados três experimentos distintos para analisar diferentes cenários de hierarquia de processos:

\begin{table}[H]
\centering
\begin{tabular}{@{}llcc@{}}
\toprule
\textbf{Experimento} & \textbf{Descrição} & \textbf{Sleep Pai} & \textbf{Sleep Neto} \\
\midrule
B - B & Experiment B - Sem espera & 0ms & 0ms \\
B - C & Experiment B - Pai espera & 100ms & 0ms \\
B - D & Experiment D - Neto órfãos & 0ms & 100ms \\
\bottomrule
\end{tabular}
\caption{Descrição dos experimentos realizados}
\label{tab:experiments}
\end{table}

Cada experimento foi projetado para investigar condições específicas:

\begin{itemize}
    \item \textbf{Experimento B}: Sem espera - Processos pais e netos executam sem delays, permitindo observar condições de corrida naturais.
    \item \textbf{Experimento C}: Pai espera - O processo pai (filho do processo principal) dorme por um período, potencialmente permitindo que o neto termine antes.
    \item \textbf{Experimento D}: Neto órfão - O processo neto dorme por um período, aumentando a chance do pai terminar antes e torná-lo órfão.
\end{itemize}

\section{Resultados}

\subsection{Processos Órfãos}

A análise dos logs mostra diferentes taxas de ocorrência de processos netos órfãos (onde PPID = 1):

\begin{table}[H]
\centering
\begin{tabular}{@{}lccc@{}}
\toprule
\textbf{Experimento} & \textbf{Órfãos} & \textbf{Total} & \textbf{Porcentagem} \\
\midrule
Experimento B & 984 & 12000 & 8.2\% \\
Experimento C & 220 & 6000 & 3.7\% \\
Experimento D & 2564 & 6000 & 42.7\% \\
\bottomrule
\end{tabular}
\caption{Processos netos órfãos por experimento}
\label{tab:orphans}
\end{table}

\begin{figure}[H]
\centering
\begin{tikzpicture}
\begin{axis}[
    ybar,
    bar width=1cm,
    xlabel={Experimento},
    ylabel={Porcentagem de Processos Órfãos (\%)},
    symbolic x coords={B, C, D},
    xtick=data,
    ymin=0, ymax=40,
    nodes near coords,
    nodes near coords align={vertical},
    enlarge x limits=0.25,
    legend style={at={(0.5,-0.15)}, anchor=north,legend columns=-1},
    ]
\addplot coordinates {(B,8.2) (C,3.7) (D,42.7) };
\end{axis}
\end{tikzpicture}
\caption{Porcentagem de processos netos órfãos por experimento}
\label{fig:orphans}
\end{figure}

\subsection{Análise dos Resultados}

Observamos padrões claros nos resultados dos experimentos:

\begin{itemize}
    \item \textbf{Experimento B (Sem espera)}: Mesmo sem atrasos explícitos, ocorreram aproximadamente 8.2\% de processos netos órfãos. Isso ilustra que condições de corrida existem naturalmente devido à programação do sistema operacional e à ordem de execução dos processos.

    \item \textbf{Experimento C (Pai espera)}: Quando o processo pai dorme, observamos aproximadamente 3.7\% de netos órfãos. Curiosamente, este valor é similar ao do Experimento B, sugerindo que o atraso no pai não aumenta significativamente a taxa de órfãos.

    \item \textbf{Experimento D (Neto órfão)}: Este experimento apresentou a maior taxa de órfãos, aproximadamente 42.7\% . Este resultado era esperado, pois quando o neto dorme, há maior probabilidade do pai terminar antes dele, tornando-o órfão.
\end{itemize}

\section{Conclusão}

Os resultados demonstram claramente como o tempo de execução dos processos afeta a hierarquia de processos e a criação de processos órfãos:

\begin{enumerate}
    \item Mesmo sem atrasos explícitos, ocorrem processos órfãos devido às condições naturais de execução.
    \item O atraso no processo pai não aumenta significativamente a taxa de processos netos órfãos.
    \item Quando o processo neto executa por mais tempo (simulado pelo sleep), a probabilidade dele se tornar órfão aumenta consideravelmente.
\end{enumerate}

Estes resultados enfatizam a importância de entender a hierarquia de processos ao desenvolver sistemas com múltiplos processos, especialmente quando há dependências entre eles ou necessidade de manter estruturas hierárquicas específicas.

\end{document}

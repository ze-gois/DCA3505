\documentclass[10pt,a4paper]{article}
\usepackage[utf8]{inputenc}
\usepackage[T1]{fontenc}
\usepackage{amsmath,amssymb}
\usepackage{graphicx}
\usepackage{listings}
\usepackage{xcolor}
\usepackage{hyperref}
\usepackage{geometry}
\usepackage{setspace}

\geometry{a4paper, margin=1.8cm}
\singlespacing

\lstset{
  basicstyle=\scriptsize\ttfamily,
  columns=flexible,
  breaklines=true,
  commentstyle=\color{gray},
  keywordstyle=\color{blue},
  stringstyle=\color{red},
  frame=single,
  numbers=left,
  numberstyle=\tiny,
  stepnumber=1,
  numbersep=5pt,
  aboveskip=1pt,
  belowskip=1pt,
}

\title{Modo Usuário e Chamadas de Sistema}
\author{Relatório da Tarefa 2}
\date{\today}

\begin{document}

\maketitle

\section{Introdução}

Este relatório analisa a interação entre programas e o sistema operacional através de chamadas de sistema, comparando implementações em C e Assembly que utilizam a chamada de sistema \texttt{write} para escrever na tela. As chamadas de sistema são o mecanismo fundamental que permite a transição segura entre o código de usuário (com privilégios limitados) e o código do kernel (com acesso completo ao hardware).

\section{Implementação em C}

O programa em C utiliza a função \texttt{write()} da biblioteca padrão:

\begin{lstlisting}[language=C, frame=single, numbers=left]
#include <unistd.h>
int main() {
  const char *msg = "Hello from C program using write syscall\n";
  write(STDOUT_FILENO, msg, 38);
  return 0;
}
\end{lstlisting}

Diretrizes à execução:

\begin{lstlisting}[frame=single, numbers=left]
# Compilacao e execucao com strace (versao C)
$ gcc -static -o write_c write_c.c   # Compilacao estatica
$ ./write_c                          # Execucao normal
$ strace ./write_c                   # Execucao com rastreamento
\end{lstlisting}

Análise com \texttt{strace} (compilação estática):

\begin{lstlisting}[frame=single, numbers=left]
execve("./write_c", ["./write_c"], 0x7ffe070c1650 /* 63 vars */) = 0
brk(NULL)                               = 0x38333000
brk(0x38333d40)                         = 0x38333d40
arch_prctl(ARCH_SET_FS, 0x383333c0)     = 0
set_tid_address(0x38333690)             = 29421
set_robust_list(0x383336a0, 24)         = 0
mprotect(0x4a2000, 20480, PROT_READ)    = 0
write(1, "Hello from C program using write"..., 38) = 38
exit_group(0)                           = ?
+++ exited with 0 +++
\end{lstlisting}

\section{Implementação em Assembly}

O programa em Assembly implementa diretamente a chamada de sistema:

\begin{lstlisting}[language={[x86masm]Assembler}, frame=single, numbers=left]
.section .data
msg: .ascii "Hello from Assembly program using syscall\n"
     .set len, . - msg
.section .text
.global _start
_start:
    mov $1, %rax       # write syscall
    mov $1, %rdi       # stdout
    lea msg(%rip), %rsi # msg ptr
    mov $len, %rdx     # length
    syscall
    mov $60, %rax      # exit syscall
    xor %rdi, %rdi     # status 0
    syscall
\end{lstlisting}

Diretrizes à execução:

\begin{lstlisting}[frame=single, numbers=left]
# Montagem, ligacao e execucao (versao Assembly)
$ as -o write_asm.o write_asm.s      # Montagem do codigo
$ ld -o write_asm write_asm.o        # Ligacao do objeto
$ ./write_asm                        # Execucao normal
$ strace ./write_asm                 # Execucao com rastreamento
\end{lstlisting}

Análise com \texttt{strace}:

\begin{lstlisting}[frame=single, numbers=left]
execve("./write_asm", ["./write_asm"], 0x7ffd182c4500 /* 63 vars */) = 0
write(1, "Hello from Assembly program usin"..., 42) = 42
exit(0)                                 = ?
+++ exited with 0 +++
\end{lstlisting}

\section{Análise Comparativa}

\subsection{Papel da Instrução syscall em Assembly}

A instrução \texttt{syscall} em arquiteturas x86-64 é responsável por:
\begin{itemize}
  \item Interromper a execucao normal do programa em modo usuario
  \item Salvar automaticamente o ponteiro de instrucao (RIP) e flags (RFLAGS) na pilha do kernel
  \item Transferir o controle para o kernel, mudando para o modo kernel (Ring 0)
  \item Carregar o endereco da rotina de tratamento de syscall do kernel a partir do MSR \texttt{LSTAR}
  \item Executar o codigo do kernel correspondente a chamada especificada no registrador RAX
  \item Retornar o controle ao programa via \texttt{sysret}, voltando para o modo usuario (Ring 3)
\end{itemize}

Durante a transição do modo usuário para o kernel, ocorre uma sequência detalhada de operações:
\vspace{-0.2cm}
\begin{itemize}\setlength\itemsep{0pt}
  \item \textbf{Salvamento automático}: O processador salva RIP, RFLAGS, RSP, CS e SS
  \item \textbf{Seletores de segmento}: CS e SS recebem valores de kernel (0x08 e 0x10 respectivamente)
  \item \textbf{Troca de pilha}: RSP é atualizado para apontar para a pilha do kernel desse processo
  \item \textbf{Desabilitação de interrupções}: Interrupções são temporariamente desabilitadas
  \item \textbf{Chaveamento de contexto de memória}: CR3 pode mudar para o espaço de endereçamento do kernel
  \item \textbf{Atualização de estruturas}: A \texttt{task\_struct} é marcada como \texttt{in\_syscall=1}
\end{itemize}
\vspace{-0.2cm}

Esta instrução é o mecanismo de baixo nível que permite a comunicação controlada entre programas de usuário e o kernel.

\subsection{Diferenças entre os Programas}
\vspace{-0.2cm}
\begin{itemize}\setlength\itemsep{0pt}
  \item \textbf{Numero de syscalls}: C faz mais chamadas relacionadas a inicializacao; Assembly usa apenas write e exit

  Análise detalhada das chamadas:
  \begin{itemize}\setlength\itemsep{-2pt}
    \item \textbf{execve}: Carrega o programa, substituindo o processo atual.
    \item \textbf{brk}: Gerencia o heap, alocando/expandindo memoria.
    \item \textbf{arch\_prctl}: Configura registrador FS para thread-local storage.
    \item \textbf{set\_tid\_address}: Define endereco para ID da thread.
    \item \textbf{set\_robust\_list}: Registra lista de mutexes do processo.
    \item \textbf{mprotect}: Marca areas de memoria como somente leitura.
    \item \textbf{write}: Nossa chamada proposital - escreve no stdout.
    \item \textbf{exit\_group}: Termina o processo com codigo 0.
  \end{itemize}

  \item \textbf{Abstracao}: C encapsula detalhes via \texttt{write()}; Assembly programa explicitamente registradores/syscall
  \item \textbf{Controle}: Assembly controla diretamente a transicao; C delega a libc
  \item \textbf{Eficiencia}: Assembly elimina overhead de camadas intermediarias
\end{itemize}
\vspace{-0.2cm}

\subsection{Necessidade de Chamadas de Sistema}
\vspace{-0.2cm}
Chamadas de sistema sao necessarias devido a arquitetura de protecao entre modo usuario e kernel:
\begin{itemize}\setlength\itemsep{0pt}
  \item \textbf{Seguranca}: Impede acesso direto ao hardware/areas criticas
  \item \textbf{Estabilidade}: Gerencia ordenadamente recursos compartilhados
  \item \textbf{Virtualizacao}: Abstrai dispositivos para multiplos programas
  \item \textbf{Controle}: Define politicas de acesso a recursos
\end{itemize}
\vspace{-0.2cm}

Para escrever na tela, o programa precisa acessar o terminal, que e um recurso compartilhado gerenciado pelo sistema operacional. Sem o controle centralizado via chamadas de sistema, haveria conflitos de acesso e possiveis instabilidades.

\subsection{Transição Segura entre Modos}

A transicao segura entre modos de execucao e garantida por:

\begin{enumerate}
  \item \textbf{Instrucoes privilegiadas}: Apenas instrucoes especificas como \texttt{syscall} podem solicitar a mudanca para o modo kernel.

  \item \textbf{Tabela de chamadas de sistema}: O kernel mantem uma tabela de funcoes permitidas (\texttt{sys\_call\_table} no Linux) que podem ser invocadas via chamadas de sistema.

  \item \textbf{Validacao de parametros}: O kernel valida todos os parametros antes de executar operacoes.

  \item \textbf{Hardware de protecao}: O MMU (Memory Management Unit) e os niveis de privilegio do processador (rings) impedem acesso nao autorizado.

  \item \textbf{Troca de pilhas}: O kernel usa uma pilha separada para evitar que o modo usuario manipule a pilha do kernel.
\end{enumerate}

\noindent Estado do processo e visão interna da transição:
\begin{itemize}\setlength\itemsep{0pt}
  \item \textbf{Registradores}: Todos os registradores de usuário são salvos na estrutura \texttt{pt\_regs} na pilha do kernel
  \item \textbf{Estruturas de controle}:
    \begin{itemize}\setlength\itemsep{0pt}
      \item \texttt{task\_struct}: Contém todo o estado do processo (PID, prioridade, CPU, etc.)
      \item \texttt{thread\_info}: Armazena informações específicas da thread atual
      \item Campo \texttt{flags}: Bits indicam \texttt{TIF\_SYSCALL\_TRACE} se strace estiver ativo
    \end{itemize}
  \item \textbf{Memoria}: Espaco de usuario permanece mapeado mas protegido (kernel pode acessar memoria do usuario atraves de funcoes como \texttt{copy\_from\_user} com verificacoes)
  \item \textbf{Fluxo de execução}:
    \begin{itemize}\setlength\itemsep{0pt}
      \item Usuário → \texttt{syscall} → \texttt{entry\_SYSCALL\_64} → \texttt{do\_syscall\_64} → \texttt{sys\_write}
      \item Retorno: \texttt{sysret} restaura contexto armazenado, devolvendo controle ao userspace
    \end{itemize}
\end{itemize}
\vspace{-0.2cm}

Se o modo usuário tivesse acesso direto aos dispositivos:

\begin{enumerate}
  \item Programas maliciosos poderiam modificar configuracoes criticas de hardware
  \item Multiplos programas tentando controlar o mesmo dispositivo causariam conflitos
  \item Nao haveria abstracao de hardware, complicando o desenvolvimento
  \item Programas poderiam acessar dados sensiveis de outros programas
  \item A estabilidade do sistema seria comprometida pela falta de coordenacao
\end{enumerate}

\section{Conclusão}

A analise das implementacoes em C e Assembly demonstra a importancia das syscalls como interface controlada entre userspace e kernel. Esta separacao com transicoes controladas via \texttt{syscall} e fundamental para a seguranca e estabilidade dos sistemas modernos.

As transicoes envolvem salvamento e restauracao precisos do estado de execucao, incluindo:
\begin{itemize}\setlength\itemsep{0em}
  \item Conteudo dos registradores (RIP, RFLAGS, registradores de proposito geral)
  \item Ponteiro da pilha e seletores de segmento
  \item Estado de privilegio (mudanca entre Ring 3 e Ring 0)
  \item Atualizacao das estruturas de controle do kernel (\texttt{task\_struct}, \texttt{thread\_info})
  \item Configuracao de protecoes de memoria para prevenir acesso nao autorizado
\end{itemize}

O processador x86-64 implementa otimizacoes especificas para syscalls, como as instrucoes dedicadas \texttt{syscall}/\texttt{sysret} e os registradores MSR (Model Specific Registers) que armazenam os pontos de entrada/saida do kernel. Esse hardware especializado torna as transicoes entre modo usuario e kernel muito mais eficientes do que interrupcoes genericas.

Todo esse mecanismo, mesmo parecendo excessivo para operacoes simples como escrever na tela, e essencial para manter o sistema operacional protegido e estavel. Ele permite que multiplos programas compartilhem recursos de forma segura e coordenada, balanceando facilidade de programacao (abordagem C) com eficiencia e controle preciso (abordagem Assembly).

\end{document}
